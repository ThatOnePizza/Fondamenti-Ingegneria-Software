Il design definisce la struttura della soluzione.

\subsection{Livelli di Design}
\begin{itemize}
    \item \textbf{Architectural design}: mappa i requisiti su architettura Software e componenti/sottosistemi
    \item \textbf{Component design}: fissa dettagli dei componenti, specificando maggiormente la soluzione
\end{itemize}

\subsection{Ri-uso}
Ci sono 3 livelli di ri-uso:
\begin{itemize}
    \item \textbf{Clonazione}: Si riutilizza interamente design/codice, con piccoli aggiustamenti
    \item \textbf{Design pattern}: Buona soluzione a problema ricorrente
    \item \textbf{Stili architetturali}: Architettura generica che suggerisce come decomporre il sistema
\end{itemize}

\subsection{Design Architetturale}
processo di design per identificare:
\begin{itemize}
    \item le (macro) componenti di un sistema
    \item il framework per il controllo e la comunicazione tra componenti
\end{itemize}

\noindent Produce una descrizione dell'architettura software 

\subsubsection{Vantaggi}
\begin{itemize}
    \item Guida lo sviluppo ed aiuta nella comprensione del sistema
    \item Documenta il sistema
    \item Aiuta a ragionare sull'evoluzione del sistema
    \item Supporta decisioni manageriali
    \item Facilita l’analisi di alcune proprietà
    \item Permette il riuso (Large-scale)
\end{itemize}

\subsubsection{Componenti}
\begin{itemize}
    \item Sottosistema: \`E un sistema da per s\`e, può essere eseguito da solo.
    \item Modulo: è un’unità del sistema che offre servizi ad altre unità, ma che non può essere considerato un sistema a se stante
\end{itemize}

\noindent Nella pratica spesso i due termini vengono usati come sinonimi
