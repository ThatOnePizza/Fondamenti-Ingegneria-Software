Esprimono i \textbf{Requisiti funzionali} di un sistema.\\
Totalmente indipendente dal mondo OO e NON visuale. Implementati dallo Use Case Diagram UML per averne una rappresentazione visuale.

\noindent Gli use case esprimono l’interazione tra le entità che interagiscono con il sistema stesso (chiamate attori)

\noindent Sempre descritti dal punto di vista dell'utente.

\subsection{Scenario}
Uno scenario è una sequenza ordinata di interazioni tra un sistema e gli attori.\\
Rappresenta una particolare esecuzione di uno use case (istanza), e rappresenta un singolo cammino dello use case

\subsection{Composto da}
\begin{itemize}
    \item \textbf{Attori}: I ruoli assunti dalle entità (persone, sistemi hardware/software, dispositivi) che interagiscono con il sistema.
    \begin{itemize}
        \item \textbf{Primari}: chi ha delle mire sul sistema o chi guadagna qualcosa dal sistema.
        \item \textbf{Secondari}: quelli su cui il sistema ha delle mire o chi produce qualcosa (o offre un servizio) per il sistema
    \end{itemize}
    
    \item \textbf{Use case}: quello che gli attori possono fare con il sistema.
    \begin{itemize}
        \item È un insieme di scenari che hanno in comune lo scopo finale dell'utente.
        \item Uno scenario rappresenta un singolo cammino attraverso lo use case.
    \end{itemize}
    
    \item \textbf{Relazioni}: tra gli attori e gli use case (indica partecipazione!).
    \item \textbf{Confini del sistema}: un rettangolo disegnato intorno agli use case per indicare i confini del sistema.
\end{itemize}

\subsection{Descrizione}
\begin{itemize}
    \item \textbf{Scenario principale}: normale flusso di eventi.
    \item \textbf{Scenari secondari}: cosa può succedere di sbagliato/diverso e come gestirlo.
\end{itemize}

\noindent Il nome va sempre in \textbf{UpperCamelCase}

\noindent Ogni interazione/passo viene definito come "$<$numero$>$ il $<$cliente/sistema$>$ $<$azione$>$"

\subsection{Gerarchie di attori}
Gli attori possono avere gerarchie, indicate da frecce le quali puntano verso l'attore dal quale di eredita.

\subsection{Relazioni tra use case}
\begin{itemize}
    \item \textbf{Inclusione}: "includere" comportamenti comuni, ad esempio un login viene incluso dagli use case di acquisto. (Freccia da acquisto a login)
    \item \textbf{Estensione}: Per specificare comportamenti opzionali, ad esempio la conferma del 2FA nel caso l'utente lo abbia attivato. (Freccia da 2FA a Login)
    \item \textbf{Generalizzazione}: Per generalizzare uno use case, ad esempio Pay generalizzato in Pay by credit e Pay by cash.
\end{itemize}
