\subsection{Modelli Waterfall}
\textbf{Pregi}:
\begin{itemize}
    \item Enfasi su aspetti come l’analisi dei requisiti e il progetto di sistema
    \item Pospone l’implementazione dopo avere capito i bisogni del cliente
    \item Introduce disciplina e pianificazione
    \item \`E applicabile se i requisiti sono chiari e stabili
\end{itemize}

\noindent \textbf{Difetti}:
\begin{itemize}
    \item Lineare, rigido, monolitico
    \item La consegna avviene dopo anni, intanto i requisiti cambiano o si chiariscono: così viene consegnato software obsoleto
    \item Viene prodotta molta “carta” poco chiara
\end{itemize}

\subsubsection{Sotto-tipi}
\begin{itemize}
    \item Modello waterfall con feedback e iterazioni
    \item Modello waterfall con prototipazione
    \item V-Model
\end{itemize}

\subsubsection{V-Model}
Associa ad ogni fase una fase di testing che in caso di fallimento porta alla ripetizione della corrispondente fase fino al superamento dei test.

\subsection{Modelli Evolutivi}
\textbf{Pregi}:
\begin{itemize}
    \item Il cliente può non sapere esattamente cosa vuole all'inizio dello sviluppo
    \item I requisiti possono cambiare nel tempo
    \item Il cliente vede delle fasi intermedie del software
    \item Riduce l'impatto di un errore nella fase iniziale
\end{itemize}

\subsection{Modelli a prototyping}
\textbf{Pregi}:
\begin{itemize}
    \item Permette di raffinare requisiti definiti in termini di obiettivi generali e troppo vaghi
    \item Rilevazione precoce di errori di interpretazione
\end{itemize}

\noindent \textbf{Svantaggi}:
\begin{itemize}
    \item il prototipo è meccanismo per identificare i requisiti, spesso da “buttare”
\end{itemize}

\subsection{Modelli iterativi incrementali}
\textbf{Pregi}:
\begin{itemize}
    \item Le release sono sviluppate considerando il feedback degli utenti
    \item E’ più semplice prevedere le risorse necessarie allo sviluppo perché ogni fase è più piccola e semplice dell'intero sistema
    \item Eventuali problemi vengono individuati presto
    \item Risposta rapida al mercato siccome abbiamo delle release intermedie
\end{itemize}

\subsubsection{Modelli UP}
Meta-modello iterativo e incrementale con diverse implementazioni.\\
Formato da:
\begin{itemize}
    \item Diverse iterazioni ognuna divisa in:
    \begin{itemize}
        \item 4 fasi (Studio di fattibilità, sviluppo, design codifica e testing, deploy), dove ogni fase può essere ripetuta più volte, a loro volta formate da:
        \begin{itemize}
            \item 5 attività: Requisiti (R), Analisi (A), Design, (D) Codifica (C) e Testing (T)
        \end{itemize}
    \end{itemize}
\end{itemize}

\subsection{Meta-modello a Spirale}
(Precede UP) Primo modello a introdurre il concetto di \textbf{rischio}. Essendo un meta-modello va integrato con un altro modello precedentemente elencato.\\
\noindent \textbf{Vantaggi}:
\begin{itemize}
    \item Adatto allo sviluppo di sistemi complessi
    \item Primo approccio che considera il rischio
\end{itemize}

\noindent \textbf{Svantaggi}:
\begin{itemize}
    \item Necessita competenze di alto livello per la stima dei rischi
    \item Richiede un’opportuna personalizzazione ed esperienza di utilizzo
    \item Se un rischio rilevante non viene scoperto o tenuto a bada 'siamo da capo …'
\end{itemize}

\break
\subsection{Model Driven Development (MDD)}
\textbf{Pregi}:
\begin{itemize}
    \item Riduce la quantità di software da scrivere
    \item Consegne più veloci
    \item Riduce i costi totali di sviluppo e i rischi
\end{itemize}

\noindent \textbf{Svantaggi}:
\begin{itemize}
    \item Sono necessari dei compromessi siccome i requisiti potrebbero non essere soddisfatti dalle componenti (convincere il cliente)
    \item Integrazione non sempre facile
    \item Spesso i componenti usati sono fatti evolvere dalla ditta produttrice senza controllo di chi li usa
\end{itemize}

\subsection{Modello Trasformazionale}
Lo sviluppo del Software viene visto come una sequenza di passi che trasformano formalmente una specifica in una implementazione.
